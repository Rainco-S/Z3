\usetikzlibrary{arrows, decorations.pathmorphing, decorations.markings, shapes.geometric, calc, positioning}

\tikzset{
    node0/.style={circle, draw, fill=white, inner sep=0pt, minimum size=1.5mm},
    node1/.style={circle, draw, fill=black, inner sep=0pt, minimum size=1.5mm},
    reoEdge/.style={->, >=latex, thin},
    reoSpout/.style={<->, >=latex, thin},
    reoLossy/.style={->, >=latex, thin, dashed},
    reoWavy/.style={->, >=latex, thin, decorate, decoration={zigzag, amplitude=1mm, segment length=1.5mm, post length=1.5mm, pre length=1.5mm}},
}

% A. Basic Connectors (Sync, Lossy, Filter)
% Parameters: #1=style(optional), #2=source end, #3=sink end, #4=label text(optional)
\newcommand{\drawLink}[4][reoEdge]{
    \draw[#1] (#2) -- node[above, font=\scriptsize] {#4} (#3);
}

% B. FIFO (with Box in the Middle)
% Parameters: #1=source end, #2=sink end, #3=text inside box, #4=extra style(optional, e.g. dashed)
\newcommand{\drawFIFO}[4][]{
    \draw[reoEdge, #1] (#2) -- node[midway, sloped, allow upside down, draw, fill=white, rectangle, minimum width = 6mm, minimum height = 2mm, inner sep=1pt, font=\tiny] {#4} (#3);
}

% C. Drain (Two Sources to One Sink)
% Parameters: #1=source end 1, #2=source end 2, #3=sink end, #4=center symbol(usually ||), #5=center symbol(usually ||)
\newcommand{\drawDrain}[5]{
    % Calculate midpoints
    \coordinate (mid1) at ($(#1)!0.4!(#2)$);
    \coordinate (mid) at ($(#1)!0.5!(#2)$);
    \coordinate (mid2) at ($(#1)!0.6!(#2)$);
    % Draw arrows pointing to midpoints
    \draw[->, >=latex, thin] (#1) -- (mid1);
    \draw[thin] (mid1) -- (mid);
    \draw[thin] (mid) -- (mid2);
    \draw[->, >=latex, thin] (#2) -- (mid2);
    % Draw center symbols
    \node at (mid) [font=\tiny, inner sep=1pt] {#3};
    \node at (mid1) [font=\tiny, inner sep=1pt] {#4};
    \node at (mid2) [font=\tiny, inner sep=1pt] {#5};
    % Actually Drain doesn't have an explicit Sink node, so #3 is just for position logic or if you want to draw only lines
}

% D. Spout (One Source to Two Sinks)
% Parameters: #1=source end, #2=sink end 1, #3=sink end 2, #4=center symbol(usually ||), #5=center symbol(usually ||)
\newcommand{\drawSpout}[5]{
    % Calculate midpoints
    \coordinate (mid1) at ($(#1)!0.4!(#2)$);
    \coordinate (mid) at ($(#1)!0.5!(#2)$);
    \coordinate (mid2) at ($(#1)!0.6!(#2)$);
    % Draw arrows pointing to midpoints
    \draw[->, >=latex, thin] (mid1) -- (#1);
    \draw[thin] (mid1) -- (mid);
    \draw[thin] (mid) -- (mid2);
    \draw[->, >=latex, thin] (mid2) -- (#2);
    % Draw center symbols
    \node at (mid) [font=\tiny, inner sep=1pt] {#3};
    \node at (mid1) [font=\tiny, inner sep=1pt] {#4};
    \node at (mid2) [font=\tiny, inner sep=1pt] {#5};
    % Actually Spout doesn't have an explicit Sink node, so #3 is just for position logic or if you want to draw only lines
}

% E. Timer (Middle with Rounded Box)
% Parameters: #1=start end, #2=end end, #3=delay time t
\newcommand{\drawTimer}[3]{
    \draw[reoEdge] (#1) -- node[draw, fill=white, rectangle, rounded corners=2pt, minimum size=3.5mm, inner sep=1pt, font=\tiny] (timerbox) {#3} (#2);
}

% F. Complex Timer (with Control Ports OFF/RESET/EXPIRE)
% Parameters: #1=start end, #2=end end, #3=delay time t, #4=control type(OFF/RST/EXP), #5=control node position(above/below)
\newcommand{\drawComplexTimer}[5]{
    % 先画基础 Timer
    \draw[reoEdge] (#1) -- node[draw, fill=white, rectangle, rounded corners=2pt, minimum size=3.5mm, inner sep=1pt, font=\tiny] (timerbox) {#3} (#2);
    % 画控制端口
    \node (ctrl) [#5=0.5cm of timerbox, font=\tiny] {#4};
    \draw[->, >=latex, dashed] (ctrl) -- (timerbox);
}


% --- 1. Basic Channels ---
\newcommand{\chanSync}[2]{\drawLink{#1}{#2}{}}
\newcommand{\chanLossySync}[2]{\drawLink[reoLossy]{#1}{#2}{}}
\newcommand{\chanSyncDrain}[2]{\drawDrain{#1}{#2}{}{}{}} % Hollow Drain
\newcommand{\chanAsynDrain}[2]{\drawDrain{#1}{#2}{$||$}{}{}} % Vertical Drain
% --- 2. FIFO Variants ---
\newcommand{\chanFifoOne}[2]{\drawFIFO{#1}{#2}{}} % Buffer size 1
\newcommand{\chanFifoN}[3]{\drawFIFO{#1}{#2}{$N=#3$}} % Buffer size n
\newcommand{\chanFifoOneE}[3]{\drawFIFO{#1}{#2}{$#3$}} % Initialized with e
\newcommand{\chanFifoNE}[4]{\drawFIFO{#1}{#2}{$#3$}} % Initialized e with size n
% --- 3. Filter & Producer ---
\newcommand{\chanFilterP}[3]{\drawLink[reoWavy]{#1}{#2}{$P(#3)$}} 
\newcommand{\chanProducerP}[3]{\drawLink{#1}{#2}{Prod: $#3$}} 
% --- 4. Spouts ---
\newcommand{\chanSyncSpout}[2]{\drawSpout{#1}{#2}{}{}{}}
\newcommand{\chanAsynSpout}[2]{\drawSpout{#1}{#2}{$||$}{}{}}
% --- 5. Probabilistic/Faulty ---
\newcommand{\chanCptSync}[3]{\drawLink{#1}{#2}{$p=#3$}} % Corrupting Sync
\newcommand{\chanRdmSync}[3]{\drawLink{#1}{#2}{#3}} % Random Sync
\newcommand{\chanProbLossy}[3]{\drawLink[reoLossy]{#1}{#2}{$p=#3$}} % Probabilistic Lossy
\newcommand{\chanFtyFifoOne}[3]{\drawFIFO{#1}{#2}{Fty:$#3$}} % Faulty FIFO
\newcommand{\chanLossyFifoOne}[3]{\drawFIFO[dashed]{#1}{#2}{Lossy:$#3$}} % Lossy FIFO
% --- 6. Timers ---
\newcommand{\chanTimert}[3]{\drawTimer{#1}{#2}{#3}}
\newcommand{\chanOFFTimert}[3]{\drawComplexTimer{#1}{#2}{#3}{OFF}{above}}
\newcommand{\chanRSTTimert}[3]{\drawComplexTimer{#1}{#2}{#3}{RST}{above}}
\newcommand{\chanEXPTimert}[3]{\drawComplexTimer{#1}{#2}{#3}{EXP}{above}}